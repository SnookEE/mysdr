\section{Power Supply} %(fold)
\label{sec:power}
    Input voltage range is designed to be 7.5V to 40V and is fed into a 5V LDO that can supply up to 
    3.5A of current.  This 5V source supplies the RF daughterboards as well as a National Semiconductor
    DC/DC multi-output power supply which then supplies the 3.3V, 2.5V, 1.8V and 1.2V outputs.
    
    The 3.3V and 1.2V outputs are from switching supplies and can supply up to 1.5A each.  The 2.5V and 1.8V
    outputs are from LDO regulators which can supply 300mA each.  Refer to Table \ref{table:currentconsumption} for 
    the distribution and approximate current consumption for that power supply.
    
    \begin{table}[here]
        \begin{center}
            \begin{tabular}{|l|l|c|c|c|c|}
                \hline
                Component   & Reference             & 1.2V      & 1.8V      & 2.5V      & 3.3V      \\ \hline 
                \hline
                \hyperref[sec:fpga]{EP3C}           & U1        & TBD       & TBD       & TBD       & TBD       \\ \hline
                \hyperref[sec:gige]{VSC8601}        & U2        & \-        & \-        & \-        & 230 mA    \\ \hline
                \hyperref[sec:adc]{AD9251}          & U3        & \-        & 96 mA     & \-        & \-        \\ \hline
                \hyperref[sec:dac]{AD9717}          & U4        & \-        & 20 mA     & \-        & \-        \\ \hline
                \hyperref[sec:dac-lpf]{SKY73202}    & U5        & \-        & \-        & \-        & 59 mA     \\ \hline
                \hyperref[sec:clock]{Si5338}        & U6        & \-        & \-        & \-        & 175 mA    \\ \hline
                \hyperref[sec:sam3u]{SAM3U}         & U7        & \-        & \-        & \-        & 100 mA    \\ \hline 
            \end{tabular}
            \caption{Power supply distribution and current consumption}
            \label{table:currentconsumption}
        \end{center}
    \end{table}
    
%(end)